%%=============================================================================
%% Samenvatting
%%=============================================================================

%% TODO: De "abstract" of samenvatting is een kernachtige (~ 1 blz. voor een
%% thesis) synthese van het document.
%%
%% Deze aspecten moeten zeker aan bod komen:
%% - Context: waarom is dit werk belangrijk?
%% - Nood: waarom moest dit onderzocht worden?
%% - Taak: wat heb je precies gedaan?
%% - Object: wat staat in dit document geschreven?
%% - Resultaat: wat was het resultaat?
%% - Conclusie: wat is/zijn de belangrijkste conclusie(s)?
%% - Perspectief: blijven er nog vragen open die in de toekomst nog kunnen
%%    onderzocht worden? Wat is een mogelijk vervolg voor jouw onderzoek?
%%
%% LET OP! Een samenvatting is GEEN voorwoord!

%%---------- Nederlandse samenvatting -----------------------------------------
%%
%% TODO: Als je je bachelorproef in het Engels schrijft, moet je eerst een
%% Nederlandse samenvatting invoegen. Haal daarvoor onderstaande code uit
%% commentaar.
%% Wie zijn bachelorproef in het Nederlands schrijft, kan dit negeren en heel
%% deze sectie verwijderen.

\IfLanguageName{english}{%
\selectlanguage{dutch}
\chapter*{Samenvatting}
\lipsum[1-4]
\selectlanguage{english}
}{}

%%---------- Samenvatting -----------------------------------------------------
%%
%% De samenvatting in de hoofdtaal van het document

\chapter*{\IfLanguageName{dutch}{Samenvatting}{Abstract}}

Blockchain is een opkomende technologie die vele toepassingen zal hebben in de nabije toekomst. Blockchain is het best te vergelijken met een gedecentraliseerde databank die werkt met een peer-to-peer syteem om data op te slaan. Dit heeft heel veel voordelen, één van de voordelen is dat het zeer veilig is als er aan bepaalde voorwaarden wordt voldaan en dat de data dan uiteraard ook zeer veilig is. Wanneer bijvoorbeeld een gecentraliseerde databank gehackt zou worden dan is de data in de databank niet meer veilig, blockchain heeft hier het voordeel te werken met verschillende nodes zodat de data zeker niet kwijt kan geraakt worden tenzij alle nodes zouden crashen. Verder is het ook niet zo simpel om data in de blockchain aan te passen eenmaal deze toegevoegd werd. Het doel van de bachelorproef zal dan ook inhouden om de werking van blockchain te verduidelijken en dit grondig te bestuderen en vervolgens ook de taken van een verzekeringsmakelaar te bekijken. Tenslotte is het doel om blockchain toe te passen op enkele taken van een verzekeringsmakelaar. Zo komt het bestuderen van unieke documenten aan bod en gaan we onderzoeken hoe we deze kunnen toevoegen aan de blockchain en bekijken welke zaken vereist zijn om het hele procces zo veilig mogelijk te maken. Verder ga ik ook bekijken aan de hand van deze studie of dit ook effectief een verbetering is dan de huidige manier van werken.
