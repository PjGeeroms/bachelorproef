%%=============================================================================
%% Inleiding
%%=============================================================================

\chapter{Inleiding}
\label{ch:inleiding}

Het doel van de bachelorproef is om blockchain toe te passen op het werk dat een verzekeringsmakelaar verricht, dus het beveiligen van unieke documenten zoals verzekeringsdocumenten en voorkomen om documenten te vervalsen of dubbele schadeclaims te verrichten. Bovendien willen we ook streven naar een betere uniformiteit en communicatie tussen verschillende partijen. Dit is dan meteen ook onze usecase.

\section{Stand van zaken}
\label{sec:stand-van-zaken}

%% TODO: deze sectie (die je kan opsplitsen in verschillende secties) bevat je
%% literatuurstudie. Vergeet niet telkens je bronnen te vermelden!
Momenteel is er nog geen centrale databank van verzekeringen en heeft elke partij zijn eigen data. Zo heeft een verzekeringsmaatschapij een databank, en verschillende andere partijen hebben opnieuw hun eigen versie. Hier kan dus een duidelijke optimalisatie worden aangebracht door al deze data toegangkelijker te maken. Het zou dus veel makkelijker worden mochten al deze partijen toegang hebben tot één en dezelfde databank en hier data kan aan toevoegen. Zo is het verder ook makkelijker om later integraties te maken mocht dit nodig zijn.


\section{Probleemstelling en Onderzoeksvragen}
\label{sec:onderzoeksvragen}


%% TODO:
%% Uit je probleemstelling moet duidelijk zijn dat je onderzoek een meerwaarde
%% heeft voor een concrete doelgroep (bv. een bedrijf).
%%
%% Wees zo concreet mogelijk bij het formuleren van je
%% onderzoeksvra(a)g(en). Een onderzoeksvraag is trouwens iets waar nog
%% niemand op dit moment een antwoord heeft (voor zover je kan nagaan).

Het de bedoeling om het gehele systeem toegankelijk te maken voor de vereiste instanties. Denk maar aan de overheid, verzekeraars, garages indien het gaat over een autoverzekering, agenten en vele meer. 

Het probleem dat zich momenteel voordoet is dat er een hele hoop documenten nodig zijn en de communicatie tussen verschillende partijen een hele tijd kan duren.

Denk maar aan de situatie wanneer er een verkeersongeval gebeurt. Er moeten een schadeforumulier worden ingevuld en beide partijen hebben een versie nodig. Vervolgens moet dit worden doorgestuurd naar hun persoonlijke verzekeraar die op hun beurt contact moeten opnemen met elkaar. Er wordt bij beide partijen dan ook een schadedocument opgemaakt. Vervolgens moet er ook een inspectie gebeuren van het voertuig om de schade op te meten. Hiervoor wordt door deze partij nog enkele documenten opgemaakt. Dit document wordt dan uiteindelijk naar beide partijen verstuurd. Al deze documenten zijn echter verspreid en worden beheerd door verschillende partijen. Hierdoor onstaat er ook heel veel communicatie en vaak ook miscommunicatie. Zo weten de slachtoffers nooit hoe ver de zaken staan en is er vaak een lange wachttijd. 

Blockchain zou hier een uistekende oplossing voor zijn om alle documenten en communicatie te verrichten op deze manier. Waaruit ook volgende onderzoeksvragen voortkomen.
\begin{itemize}
	\item Hoe kan men blockchain toepassen op verzekeringsdocumenten en een eco-systeem opbouwen voor meerdere partijen?
	\item Hoe kan men blockchain gebruiken om oplichting van verzekeringen tegen te gaan.
\end{itemize}

\section{Opzet van deze bachelorproef}
\label{sec:opzet-bachelorproef}

%% TODO: Het is gebruikelijk aan het einde van de inleiding een overzicht te
%% geven van de opbouw van de rest van de tekst. Deze sectie bevat al een aanzet
%% die je kan aanvullen/aanpassen in functie van je eigen tekst.

Deze bachelorproef is verder als volgt opgebouwd:

Allereerst wordt er in Hoofdstuk \ref{ch:blockchain} een literatuurstudie gehouden over blockchain.

Vervolgens wordt er in Hoofdstuk \ref{ch:insurance} een literatuurstudie gehouden over de taken van een verzekeringsmakelaar en over verschillende soorten verzekeringen. 

In Hoofdstuk~\ref{ch:methodologie} wordt de methodologie toegelicht en worden de gebruikte onderzoekstechnieken besproken om een antwoord te kunnen formuleren op de onderzoeksvragen.

In Hoofdstuk \ref{ch:add-to-blockchain} wordt onderzocht hoe data kan worden toegevoegd aan een blokchain. Verder wordt onderzocht welke zaken van de transactie afgeschermd moet worden om breuken op de privacy te voorkomen. 

In Hoofdstuk \ref{ch:eco-system} wordt een systeem gebouwd aan de hand van de verschillende types van blockchain. Hier worden meteen ook de mogelijke algoritmen toegepast en besproken  en vervolgens ook vergeleken met elkaar. Verder worden de voor- en nadelen meteen weergegeven. 

In Hoofdstuk \ref{ch:usecases} worden enkele usecases opgelost met blockchain en worden duidelijk vereenvoudigd door gebruik van blockchain. 

In Hoofdstuk \ref{ch:alternative-technology} wordt vervolgens een alternatieve technologie besproken en wordt onze onderzoeksvraag uitgewerkt met deze technologie. Ook van deze technologie zullen de voordelen en eventuele nadelen aangekaart worden. 

In Hoofdstuk~\ref{ch:conclusie}, tenslotte, wordt de conclusie gegeven en een antwoord geformuleerd op de onderzoeksvragen. Daarbij wordt ook een aanzet gegeven voor toekomstig onderzoek binnen dit domein.

