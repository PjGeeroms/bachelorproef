%%=============================================================================
%% Conclusie
%%=============================================================================

\chapter{Conclusie}
\label{ch:conclusie}

%% TODO: Trek een duidelijke conclusie, in de vorm van een antwoord op de
%% onderzoeksvra(a)g(en). Wat was jouw bijdrage aan het onderzoeksdomein en
%% hoe biedt dit meerwaarde aan het vakgebied/doelgroep? Reflecteer kritisch
%% over het resultaat. Had je deze uitkomst verwacht? Zijn er zaken die nog
%% niet duidelijk zijn? Heeft het onderzoek geleid tot nieuwe vragen die
%% uitnodigen tot verder onderzoek?

Na het overlopen van de werking van blockchain en zijn verschillende varianten alsook het overlopen van de werking van hashgraph is er duidelijk een verbetering door het toepassen van beide technologiën. Dit resultaat was ook te verwachten voor de start van dit onderzoek omdat dit ook de perfecte technologie is voor nog veel meer problemen op te lossen waar verschillende partijen toegang moeten hebben tot bepaalde data. 

Zowel blockchain als hashgraph zijn zeker toepasbaar op de onderzoeksvraag. Op de datum van het schrijven van deze bachelorproef heeft blockchain de voorkeur door het lange bestaan en de populariteit. Hashgraph zelf is opgedoken in 2016 en dus ook nog jong. Het is zeker een technologie die blockchain kan overtreffen door het sneller bereiken van een overeenkomst wat een voordeel is voor systemen die snel moeten werken. Echter is er voor hashgraph nog geen publiek netwerk opgesteld zoals Ethereum waardoor hashgraph momenteel vooral gebruikt wordt in een beperkt netwerk. Hierdoor valt de mogelijkheid momenteel ook weg om een hashgraph op te stellen op een publiek netwerk voor alle partijen.

Elk type blockchain kan dus gebruikt worden om onze onderzoeksvraag op te lossen maar er wordt niet aangeraden om een openbaar netwerk te gebruiken zonder dit af te schermen aangezien het toch om zeer gevoelige data gaat. Daarom werd er in de usescases in Hoofdstuk \ref{ch:usecases} ook gebruik gemaakt van een consortium blockchain. Zo kan er gebruik gemaakt worden van smart contracts en kunnen alle voordelen van een publieke blockchain en een privé blockchain gebruikt worden. Ook zorgt ``multi-sig'' er voor dat transacties vrij snel verwerkt worden en er snel een overeenkomst kan worden bereikt.

Wanneer we onze opstelling zouden plaatsen op een beperkt netwerk en slechts enkele partijen toegang zouden geven tot dit netwerk zou hashgraph zeker het meest optimaal werk leveren maar door het gebruik van consortium blockchain en multi-sig zal de snelheid ongeveer gelijk zijn. Dat hashgraph sneller zou zijn dan blockchain is ook nog nergens bewezen, dit wordt uit theorie verondersteld door het gebruik van de kleine data transfers die hashgraph gebruikt. 

\section{Hoe kan men blockchain toepassen op verzekeringsdocumenten en een eco-systeem opbouwen voor meerdere partijen?}
De meest optimale manier voor het opstellen van een eco-systeem waar in dit geval met gevoelige informatie wordt omgegaan is het aangeraden om met een consortium blockchain te werken. Zo zijn alle nodes toch gekend en kan er gebruik gemaakt worden van handige technieken zoals smart contracts wanneer met dit eco-systeem bouwt op een Ethereum blockchain. Door aan deze blockchain de nodige partijen toe te voegen kan elke partij dezelfde data gebruiken. Elke nieuwe verzekering die wordt afgesloten wordt meteen toegevoegd aan de blockchain. Op deze manier is het later ook makkelijk wanneer er een ongeval gebeurd dat dit meteen kan toegevoegd worden aan de blockchain en alle data makkelijk te verkrijgen is voor verschillende instanties. Documenten kunnen ook niet meer worden veranderd nadien eenmaal deze zijn toegevoegd aan de blockchain. Hierdoor voorkomen we oplichting of vervalsing van documenten aangezien elk document in de blockchain uniek zal zijn door het gebruik van een hash.

Wanneer er later controles moeten uitgevoerd worden is het op deze manier ook eenvoudiger om dit te doen aangezien alle data dat met deze persoon of verzekering te maken heeft gecentraliseerd is en ook beschikbaar is voor andere instanties.

\section{Hoe kan men blockchain gebruiken om oplichting van verzekeringen tegen te gaan.}

Alleen door het gebruik van blockchain wordt dit probleem al opgelost. Om ons voorbeeld nemen van het proberen dubbel innen van een schadeclaim dan is dit snel opgemerkt door de blockchain te raadplegen. Wanneer men de blockchain zelf zou proberen te manipuleren of met andere woorden zou proberen ``double spending'' toe te passen dan zou deze hier nog steeds niet in slagen aangezien enkel de maker van de blok dit zou kunnen en dit dus met grote kans een overheidsinstantie zal zijn. Op deze manier is het proberen vervalsen van documenten al meteen een heel stuk moeilijker geworden en is vervalsing ook meteen sneller na te gaan door de blockchain te consulteren.



%%\lipsum[76-80]

