%%=============================================================================
%% Voorwoord
%%=============================================================================

\chapter*{Voorwoord}
\label{ch:voorwoord}

%% TODO:
%% Het voorwoord is het enige deel van de bachelorproef waar je vanuit je
%% eigen standpunt (``ik-vorm'') mag schrijven. Je kan hier bv. motiveren
%% waarom jij het onderwerp wil bespreken.
%% Vergeet ook niet te bedanken wie je geholpen/gesteund/... heeft

\section*{Waarom?}
Ik heb het onderwerp ``blockchain'' gekozen als onderwerp door mijn co-promotor (Wim De Vos). Ik zelf kende het onderwerp nog niet en heb dit eigenlijk via hem vernomen en een korte inleiding over gekregen. Ik vond dit onderwerp dan ook heel interessant door de vele mogelijke toepassingen. Voorheen werd alles opgeslaan in één centrale databank maar dat is met blockchain nu helemaal anders aangezien iedereen een kopië kan krijgen van de databank. Uiteraard zijn persoonlijke gegevens versleuteld en zijn bepaalde zaken niet zichtbaar. Als we Bitcoin als voorbeeld nemen dan is bijvoorbeeld een transactie tussen twee personen wel te zien maar kan men niet zien wie deze twee individuen juist zijn. Ik begon al meteen zelf na te denken waar deze technologie nog van toepassing zou kunnen zijn en moest meteen denken aan zaken zoals notarissen en verzekeringsmakelaars die unieke documenten moeten opstellen. Zo kan een akte bijvoorbeeld in een blockchain opgenomen worden en is dit document meteen ook uniek aangezien het niet kan aangepast worden of de hele ketting herbouwd moet worden. Dit is ook meteen hetzelfde voor een verzekeringsmakelaar, dit zou bijvoorbeeld voorkomen dat individuen een dubbele schadeclaim opeisen en dergelijke. Bovendien is het ook veilig aangezien de data verspreid is over meerdere nodes of servers zoals later wordt uitgelegd.

\newpage
\section*{Dankwoord}
Graag zou ik zeker Wim De Vos willen bedanken, hij heeft mij zeer goed gesteund tijdens deze bachelorproef en bij het vinden van een goede sector om deze technologie op toe te passen. Zo gaf hij mij het voorbeeld van een notaris. Verder heeft hij mij ook in contact gebracht met Pieter Noyens die zelf een blockchain programmeur is.

Pieter Noyens heeft mij dan ook heel geholpen met het volledig begrijpen van het onderwerp aangezien het toch een vrij complexe technologie is en het overlopen van manieren om dit toe te passen op de gekozen sector. 

Verder wil ik ook Audry Vanderstraeten bedanken, via hem heb ik ook Wim leren kennen toen ik begonnen ben bij Digital Leader als jobstudent en voor de mooie kans die mij gegeven werd om in een leuke startup te beginnen.

Als laatste wil ik zeker Antonia Pierreux bedanken voor de uitstekende begeleiding tijdens de stageperiode en het schrijven van de bachelorproef. Zo heeft ze mij altijd goed bijgestaan en altijd goede en correcte feedback gegeven op alles wat ik deed. Bij het bekijken van mijn werk liet ze mij dan ook telkens wat verder denken over zaken die ik op een bepaalde manier had opgelost. Ze stond telkens paraat voor haar studenten en dat mag zeker ook eens gezegd worden. Wat ik vooral van haar geleerd heb? Flexibel zijn! Dit heb ik ook ondervonden dat je op alle vlakken wel flexibel moet kunnen zijn zowel tijdens het schrijven van de bachelorproef als bij het uitvoeren van de stage.